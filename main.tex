\documentclass[a4paper, 12pt]{article} % тип документа

% отступы
\usepackage[left=2cm,right=2cm,top=2cm,bottom=3cm,bindingoffset=0cm]{geometry}

% русский язык
\usepackage[T2A]{fontenc} % кодировка
\usepackage[utf8]{inputenc} % кодировка исходного кода
\usepackage[english,russian]{babel} % локализация и переносы

% вставка картинок
\usepackage{wrapfig}
\usepackage{graphicx}
\graphicspath{{pictures/}}
\DeclareGraphicsExtensions{.pdf,.png,.jpg}
\usepackage{caption}
\usepackage{float}

% оглавление
\usepackage{titlesec}
\titlespacing{\chapter}{0pt}{-30pt}{12pt}
\titlespacing{\section}{\parindent}{5mm}{5mm}

% межстрочный интервал
\usepackage{setspace}
\setstretch{1.2}

\usepackage{hyperref}


% графики
\usepackage{multirow}
\usepackage{pgfplots}
\pgfplotsset{compat=1.9}

% математика
\usepackage{amsmath, amsfonts, amssymb, amsthm, mathtools}

% списки
\usepackage{enumitem}
\setlist[itemize]{itemsep=1pt, topsep=1pt}


\title{\textbf{Пояснительная записка к проекту Умная Мусорка}}

\date{}

\begin{document}

\maketitle

\section{Команда}

\subsection*{Манро Эйден}

\noindent\textbf{Роль:} Инженер-электроник, инженер-программист

\noindent\textbf{Вклад:}  
\begin{itemize}
    \item Разработка программного обеспечения устройства
    \item Проектирование и сборка электрической схемы
\end{itemize}

\subsection*{Мотыгуллин Булат}

\noindent\textbf{Роль:} Системный архитектор, инженер по технической документации

\noindent\textbf{Вклад:}  
\begin{itemize}
    \item Интеграция всех компонентов и сборка устройства
    \item Подготовка технической документации проекта
\end{itemize}

\subsection*{Вехов Владимир}

\noindent\textbf{Роль:} Инженер-конструктор, инженер-технолог

\noindent\textbf{Вклад:}  
\begin{itemize}
    \item Проектирование и изготовление корпуса устройства
    \item Монтаж и пайка электронных компонентов
\end{itemize}

\noindent Следует отметить, что указанные роли отражают ключевые направления деятельности участников. На всех этапах разработки проекта члены команды тесно взаимодействовали между собой, в равной степени внося вклад в реализацию функционала, устранение технических проблем и достижение общего результата.

\section{Причины Выбора Проекта}

Проект «Умная мусорка» выбран за его практическую ценность и инженерную насыщенность. Устройство с датчиком движения обеспечивает гигиеничное бесконтактное использование, а компактные габариты и подключение по Micro USB делают его удобным и универсальным. Мусорку можно размещать прямо на рабочем столе или использовать в любом месте — она питается от ноутбука или любого другого USB-устройства, не требуя отдельного источника питания.

\section{Цель и Задачи}

\noindent\textbf{Цель:} Разработать компактную и гигиеничную мусорку с автоматическим открытием, обеспечивающую удобный доступ к утилизации отходов в любом рабочем пространстве и предотвращающую скопление мусора вокруг.

\noindent\textbf{Задачи:}
\begin{itemize}
    \item Разработать концепцию устройства и подобрать необходимые компоненты
    \item Спроектировать и изготовить печатную плату
    \item Смонтировать и припаять компоненты на плату
    \item Спроектировать и изготовить корпус
    \item Выполнить окончательную сборку устройства
    \item Подготовить техническую и презентационную документацию проекта
\end{itemize}

\section{Описание устройства}

\noindent\textbf{Размеры устройства:} 20 x 25 x 20 см \\
\noindent\textbf{Материал корпуса:} Дерево, выбранное за его экологичность, долговечность и эстетические качества. \\
\noindent\textbf{Сценарий использования:} Для активации устройства достаточно провести рукой над датчиком движения, после чего крышка автоматически откроется. Через 5 секунд крышка закрывается автоматически, обеспечивая гигиеничность и удобство использования.

\noindent\textbf{Схема платы:}
\begin{figure}[H]
    \centering
    \includegraphics[width=0.8\textwidth]{scheme.jpeg}
    \caption{Схема печатной платы устройства}
\end{figure}


\section{Описание процесса решения поставленных задач}

Все материалы можно найти в репозитории на \href{https://github.com/aidenfmunro/SmartBin}{GitHub}, посвященный нашему проекту.

\subsection*{Разработка концепции устройства и подбор необходимых компонентов}

Умная мусорка решает несколько ключевых задач:

\begin{itemize}
    \item \textbf{Детектирование движения:} Для обеспечения бесконтактного открытия крышки необходимо точно обнаруживать приближение пользователя. Для этой цели был выбран ультразвуковой дальномер HC-SR04, который позволяет эффективно измерять расстояние с высокой точностью и при этом является недорогим и простым в использовании.
    \item \textbf{Открытие крышки:} Для управления крышкой требуется мотор, который обеспечит надёжное и быстрое открытие при минимальных энергозатратах. В качестве привода был выбран сервопривод MG90S. Этот сервопривод сочетает в себе компактные размеры, высокую мощность и энергоэффективность, что идеально подходит для нашего устройства.
    \item \textbf{Компактный контроллер:} Для обработки сигналов и управления всеми компонентами требуется компактное и мощное вычислительное устройство. В качестве центрального контроллера выбран Arduino Nano 3.0, который отличается малым размером, хорошей совместимостью с различными модулями и широким сообществом поддержки, что упрощает разработку и отладку проекта.
\end{itemize}

\subsection*{Проектирование и изготовление печатной платы}

После первоначального тестирования компонентов на макетной плате мы спроектировали печатную плату в Giga EDA и напечатали её.

\begin{figure}[H]
    \centering
    \includegraphics[width=0.65\textwidth]{plata.jpeg}
    \caption{Печатная плата, спроектированная в Easy EDA}
\end{figure}

\subsection*{Монтаж и припайка компонентов}

Собрали итоговую схему на изготовленной плате.

\begin{figure}[H]
    \centering
    \includegraphics[width=0.4\textwidth]{final_scheme.jpeg}
    \caption{Готовая схема}
\end{figure}


\subsection*{Проектирование и изготовление корпуса}


Также подготовили отсек для платы

\begin{figure}[H]
    \centering
    \includegraphics[width=0.4\textwidth]{korob.jpeg}
    \caption{Коробка для схемы}
\end{figure}

Спроектировали и изготовили деревянный корпус

\begin{figure}[H]
    \centering
    \includegraphics[width=0.4\textwidth]{korpus.jpeg}
    \caption{Корпус мусорки}
\end{figure}

\subsection*{Окончательная сборка устройства}

Собрали все компоненты воедино 

\begin{figure}[H]
    \centering
    \includegraphics[width=0.3\textwidth]{done.png}
    \caption{Итоговое устройства}
\end{figure}

\section{Анализ аналогов}

\textbf{Коммерческие аналоги:}
\begin{itemize}
    \item \textbf{Мусорное ведро Daris} (1800~руб.) — объём 12~л. Устройство ориентировано на офисное использование, обладает значительными габаритами и неудобно в использовании на рабочем столе.
    
    \item \textbf{Сенсорное бесконтактное ведро Pioneer} (7500~руб.) — объём 42~л. Значительно дороже и также не предназначено для настольного применения из-за крупных размеров.
\end{itemize}

Большинство коммерческих решений обладают избыточным объёмом — минимальный найденный вариант имел объём 9~л. В контексте портативности и удобства эксплуатации на рабочем месте ни одно из устройств не может сравниться с нашим решением.

\subsection*{Open-source решения}

Наш проект ориентирован на DIY-подход и предоставляет возможности для модификации и адаптации под конкретные нужды пользователя. Среди аналогичных open-source проектов можно выделить:

\begin{enumerate}
    \item \textbf{Smart Trash Can от AlexGyver} — реализует схожую концепцию с использованием Arduino и ультразвукового датчика расстояния.
    \item \textbf{DIY Smart Trash Bin на Instructables} — проект с применением микроконтроллера ESP8266 для организации Wi-Fi подключения.
\end{enumerate}

Таким образом, наш проект сочетает в себе компактность, доступность, а также открытость к доработке и кастомизации.

\section{Тестирование продукта}

В процессе тестирования была выявлена конструктивная проблема в механизме открывания крышки. Жёсткая связь между сервоприводом и крышкой оказалась неэффективной из-за несовпадения осей вращения, что вызывало ограничение хода и нестабильную работу привода. Проблема была решена путём замены жёсткого соединения на гибкий элемент, компенсирующий разницу в траекториях движения.

По результатам эксплуатации устройства в повседневных условиях можно сделать следующие выводы:
\begin{itemize}
    \item Механизм работает стабильно, без зависаний и сбоев.
    \item Программное обеспечение функционирует корректно, утечек памяти не наблюдается.
    \item Устройство не перегревается, его температура остаётся в пределах комнатной.
    \item Скорость реакции на жесты стабильна и не снижается со временем.
    \item Энергопотребление устройства остаётся на низком уровне, что обеспечивает продолжительную автономную работу при питании от ноутбука или внешнего аккумулятора.
\end{itemize}




\end{document}
